\documentclass[11pt]{article}

\usepackage{fullpage}
%\usepackage{upgreek}
\usepackage[no-math]{fontspec}
\setmainfont[Ligatures=TeX]{Garamond}
\usepackage{wasysym}
\usepackage{cite}
%\usepackage{amsmath} % for \dfrac
%\usepackage{nopageno} % what it says on the box
%\usepackage{xltxtra} % for text super/sub scripts, but loaded by fontspec package anyway
%\usepackage{graphicx} % for including pictures, see http://en.wikibooks.org/wiki/LaTeX/Importing_Graphics
\usepackage{booktabs} % for nice tables

\begin{document}
\title{Halcyon Engine}
\date{}
\maketitle{}

\section{Introduction}
The major design goal driving the development of the Halcyon Engine is the ability to design a story-driven RPG experience while maintaining the option for the end-user to easily modify the game campaign with a minimum of programming experience.

To that end, most of the specification of games themselves is done through user-supplied files with a custom markup.
On engine start, the user-supplied files are read in and used to create the world map, NPCs, NPC attributes, and story events.

\section{Halcyon: A Series of \strikethrough{Tubes}Nodes}
The basic conceptual unit used to specify a game for the Halcyon engine is a node.
A node is delimited by curly brackets and contains several key:value pairs of data which are interpreted by the game engine.

There are several different node types, each with their own required contents.
Those types are: Story, Map, NPC, and Player.
The NPC and Player node types are identical in terms of required contents; the separate designation is so that the player character can be easily identified if many different NPC types are specified.

\subsection{Map}
hi
\subsection{Story}
nothing is here yet
\subsection{NPC and Player}
neat game though huh?
\section{Settings}
you can change them.
\end{document}