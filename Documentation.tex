\documentclass[11pt]{article}
\usepackage[normalem]{ulem}
\usepackage{fullpage}
%\usepackage{upgreek}
\usepackage[no-math]{fontspec}
\setmainfont[Ligatures=TeX]{Garamond}
\usepackage{wasysym}
\usepackage{cite}
%\usepackage{amsmath} % for \dfrac
%\usepackage{nopageno} % what it says on the box
%\usepackage{xltxtra} % for text super/sub scripts, but loaded by fontspec package anyway
%\usepackage{graphicx} % for including pictures, see http://en.wikibooks.org/wiki/LaTeX/Importing_Graphics
\usepackage{booktabs} % for nice tables

\begin{document}
\title{Halcyon Engine Documentation}
\date{}
\maketitle{}

\section{Introduction}
The major design goal driving the development of the Halcyon Engine is the ability to design a story-driven RPG experience while maintaining the option for the end-user to easily modify the game campaign with a minimum of programming experience.

To that end, most of the specification of games themselves is done through user-supplied files with a custom markup.
On engine start, the user-supplied files are read in and used to create the world map, NPCs, NPC attributes, and story events.
My goal is to define as concrete a distinction between the game's content and the way the game is handled.
Some aspects of the game files will necessarily include a small amount of Python code, and there will be a requirement that the names of the game file elements follow a prescribed format, spelling, and capitalization, but the game files should be flexible outside of those requirements.

If a player wishes to add a new location to the game's map, or add a new item to the game, or a new character to the roster, the goal is for the user to be able to do so with minimal effort and programming experience (and, unavoidably, a bit of bookkeeping).
\section{Halcyon: A Series of \sout{Tubes}Nodes}
The basic conceptual unit used to specify a game for the Halcyon engine is a node.
A node is delimited by curly brackets and contains several key:value pairs of data which are interpreted by the game engine.

There are several different node types, each with their own required contents.
Those types are: Story, Map, NPC, and Player.
The NPC and Player node types are identical in terms of required contents; the separate designation is so that the player character can be easily identified if many different NPC types are specified.

\subsection{Map}
The Map nodes define the environment in which the player moves.
Imagine a simple case of a grid of rooms, each with a door on each wall, leading to the adjacent room.
Each room would be an independent Map node, and each Map node keeps track of the NPCs currently inside it, any event flags that are tied to that room specifically, its own name and description, and the routes to other rooms.
\subsubsection{Nodes do not reflect physical reality}
However, because nodes are defined independently and only connected to each other by listing the permissible exit routes from each node, the game's map does not necessarily have to be a square grid with each room equidistant from each other, and you don't have to respect requirements of space.
For example, imagine a case with two separate nodes (A and B), each with a connection to a third node (C).
If all three nodes were plotted together on a map, it could easily be the case that each node is located on a different edge of the map (A at the far north, B far to the east, and C to the south) and there is a great number of nodes inbetween them.
This wouldn't matter, because what defines movement is the connection between nodes and not an underlying physical reality.
For an example of the utility this serves, this could be used to create a fast-travel system (the A,B,C example) or to create one-way doors (define a connection from A to B, but not from B to A).

\subsubsection{Map nodes like to talk about themselves}
Each Map node will come with its own default description as well as a list of associated story elements.
The default description is provided so that if you simply want to define a spaceholder room or you want a room for worldbuilding purposes (the oppressive waiting room you pass through before seeing the crime boss, perhaps), but don't want to go to the hassle of defining a whole set of story elements for it.
This is also used as the default description in case the story element logic breaks down somehow.
When the player enters a map node, the game will check all the associated story elements to determine if they need to be displayed and if the player meets the requirements for viewing them.
If no story elements are available or the requirements for the story elements are not met, then the default map node description will be displayed.

When the Map node file is read in, all the contents of a field are treated as a single line.
For that reason, there is some unique markup defined for the display of text within the game.
To specify line breaks, use four asterisks (****).
A single **** signal will tell the game to put a line break in and start a new paragraph with a gap between them.
It's the equivalent of hitting enter twice in a text editor, basically.

To signal a page break, use four equal signs (====).
A ==== signal will direct the game to stop rendering the text and wait for user input to continue, and when the user indicates that they're ready to move on, the display will be blanked and text will begin anew.
This is not the most useful thing for basic map descriptions but it will become important for story elements and NPC interactions.
However, it still may be useful for Map nodes.
\subsubsection{Maps are an exercise in semaphore}
flags
\subsection{Story Elements}
Story Elements are where the meat of the story and setting of the game are defined.
They are where things like conversations with NPCs are stored.
Unfortunately, they are also one example of a place where a certain amount of Python native code must be used.
However, it should not be necessary to do anything more complicated than perform a simple conditional check or call another story element's display method.

Each story element consists of several key features:
\begin{description}
	\item[Node_Name] The name of the map node which this story element belongs to. If you specify multiple node names, separated by commas, this element can start in multiple nodes. This is only for defining nodes in which this element can start autonomously; it is possible to start a story element at any time by doing things like tying it to item usage or combat outcomes.
	\item[Alternative Paths] If you want to define multiple different ways a single conversation or description can play out, one way to achieve this is by using story elements with multiple possible paths. The syntax for this requires defining a set of three fields for each pathway: Prerequisites, Story_Content, and Effects. These three fields are described in further detail below. Alternate pathways are defined by giving them a unique name inside the story element, and using the element form of [element type: path name : element contents]. The Prerequisites field will be used to determine which path is displayed.
	\item[Prerequisites] These are the things which must be the case for the story element to be displayed.
	\item[Story_Content] This is the actual text which will be displayed.
	\item[Effects] This is where you define the effects of your story. These can be things like applying stat changes, altering the player's inventory, or triggering additional story elements.
	\item[Keys] This is how you define which keys will be used to interact with the story element, such as whether it's just a simple ``yes I've read this'' or actually a choice between several dialogue options. If you want to have different options create different outcomes, you'll need to make the story element's effect tie into a separate new story element. This might be a really clunky way to handle it though.
	
\end{description}

When a story element is displayed, the player is prompted for input.
This can be either a simple Confirm command, or a Yes/No answer.

When a story element is defined, you have the option of displaying either a single default text or checking for conditionals based on the player's attributes or game flags.
For example, if you want to vary the text which is displayed 

A story element will contain its own internal flag for whether it has been seen.
If you want to define a single-instance event, then the 

A story element can also redefine the display of the buttons.
Any button which not currently have a use will be displayed as grayed-out and textless.
If you want to make it active, you need to specify it.
For example, the following: \textbf{[Keys:Commit,default;Option1,Give up]} would tell the game to display the Commit button with the default text (which is something like ``Accept''), and to enable the Option1 button with the text ``Give up''.
The story element triggered by ``Give up'' will be defined in the Effects field for this element.
\subsection{NPC and Player}
neat game though huh?
\section{Settings}
Basic game settings are defined in the settings.py file.
This is one of the places where, for ease and utility, I will have to insist on straight Python code rather than interpreted text.
However, all we're doing here is setting variable values, so it's not onerous.
This file contains settings for game size, UI colors, font size and type, and text display.
These are settings which are important but rarely need to be considered, so for the sake of readability in the main code they're kept in their own file.

I'm not using an ini file because as far as I can tell the libraries for ini interpretation require you to write out the names of the variables you're setting in your main code anyway and just refer to the ini for the corresponding value.
That's certainly a much safer way to do it, since the way I'm doing it I'm risking the possibility of injecting malicious code through the settings file, but let's be honest here any time you run someone else's code you're taking that risk and someone could just as easily have put it in the main code itself (more easily in fact, because there's more content to obscure it).

\section{Function Documentation (selected)}
Important functions and class methods will be documented in this section.
A short description of their intended use, their inputs, and outputs (if applicable) will be given.
\subsection{Screen}
\begin{description}
	\item[.draw_self(text_to_draw)] Passes a piece of text to the Screen object, which typesets it, fills in variable text, draws the Screen on the display device, and tells the game that something has updated.
\end{description}


\subsection{Story Element}
\begin{description}
	\item[.get_story_text() -> [text_to_display, effects_to_enact, buttons_to_display] or None] Chooses from the available story paths contained in the story element by evaluating each path's prerequisites. Returns None if no path is available, or a three element list.
	\item[.make_lists()] When the Story Element is first loaded from the game files, the paths are specified as single strings. This function turns the strings into lists or dictionaries as appropriate, and should be called immediately after populating the object.
\end{description}

\subsection{Map}
\begin{description}
	\item[.make_lists()] See the entry for Story Elements.
	\item[__eq__(target) -> True or False] Asks the current Map element to compare itself to the target name you provide, which should be the Node_Name attribute of another Map element. Returns True if the attributes match, False if they do not.
\end{description}

\subsection{Global functions}
\begin{description}
	\item[parse_node_file(filename, node_storage, node_type)] Used to parse a game file and store its contents as code objects. Requires you to specify the node type (Story, Map, etc.) so that required attributes can be set. Also requires you to specify a list in which to store the Node objects once they are created.
	\item[set_up_all_buttons(dict_of_buttons_to_change)] This function takes the button dictionary created by a Story Element, which is composed of key names and strings, and tells the corresponding buttons what text to display.
\end{description}
\end{document}